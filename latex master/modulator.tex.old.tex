
%Ultimate Formal Report Template 		Author: Dylan Morano ©
%
%Feel free to edit and redistribute 
%
%-------------------------------Document Information-------------------------------%

\newcommand{\Report}		{OCE-408 Final Project}			%Report Name
\newcommand{\Author}		{Dylan Morano, Rebecca Cressman, Arielle De Souza, Scott Hara}		%Author
\newcommand{\Last}		{Morano, Cressman, De Souza, Hara}				%Authors Last Name
\newcommand{\Class}		{OCE-311}			%Class Title
\newcommand{\Professor}	{Jason Dahl}			%Professor(s)

%----------------------------------------Preamble---------------------------------------%

\documentclass[10pt,letterpaper,titlepage]{report}
\usepackage[toc,page]{appendix}	%appendix support
\usepackage{fixltx2e}	%official latex patch and fixes
%\usepackage[latin1]{inputenc} 	%accepts different input encodings
\usepackage{pdflscape}	%landscape support
\usepackage{multicol}	%multicolumn support
\usepackage{setspace}	%set line spaceing (double single half)
\usepackage{geometry}	%change page geometry (margins)
\usepackage{datetime}	%automatic date/time insertion
\usepackage{fancyhdr}	%fancy headers
\usepackage{titlesec}	%Select alternative section titles
\usepackage{hyperref}	%support hypertext referencing
\usepackage{paralist} 	%enumerate and itemize within paragraphs
\usepackage{tabu}		%flexible latex tabulars
\usepackage{amsmath}	%facilitates math formulas and equations
\usepackage{amsfonts}	%math fonts and symbols
\usepackage{amssymb}	%more symbols
\usepackage{graphicx}	%enhanced support for graphics
\usepackage{subfigure}	%multiple figure containment
\usepackage{caption}	%figure captions
\usepackage{float}		%float figures within text
\usepackage{epstopdf}	%.EPS file support
% \usepackage{subfig}		%subfigures
% \usepackage{subcaption}	%subcaptions
\usepackage{listings}	%code and listing input
\usepackage{mcode} 	%MATLAB code parsing mcode.sty required in working directory
\usepackage{color} 		%custom color packaging
\usepackage{apacite}	%citing APA format

%	Define margins
\geometry{top = 1.0in, bottom = 1.0in, left = 1.0in, right = 1.0in}

%	Double spacing
\doublespacing

%	\Hide command for hiding section tidles
\newcommand*\Hide{
\titleformat{\chapter}[display]
  {}{}{0pt}{\bf \Huge}
\titleformat{\part}
  {}{}{0pt}{}
}

%	Define colors for matlab insertion
\definecolor{mygreen}{RGB}{28,172,0}  
\definecolor{mylilas}{RGB}{170,55,241}

%	define default path to graphic files
\graphicspath{ ../figures}


%	Declare types of images
\DeclareGraphicsExtensions{.pdf,.png,.jpg,.eps}


%------------------------------------------------------Begin Document------------------------------------------------------%

\begin{document}

\title{\Report}
\date{\today}
\author{\Author \\ \Class \\ \Professor}	

\pagenumbering{arabic}
\rhead{\Last}
\lhead{\Report}
\pagestyle{fancy}

%	print titlepage
\maketitle

%	print Abstract
% \section*{Abstract}
% \newpage

%	print table of contents
\tableofcontents

%	print lists of content
\listoffigures
%\listoftables
%\listofequations

%--------------------------------Begin Sections--------------------------------%
\newpage
\Hide
\chapter{Introduction}

Narragansett Town Beach is an important economic resource to the town of Narragansett, providing an average net income of $\$$270,000 each season and increasing seasonal business for surrounding restaurants, shops, and rentals. It is important to understand the natural forces which contribute to beach erosion for expensive re-nourishment project in order to keep the beach healthy for vacationers. In order to understand these processes, a study was conducted utilizing data from the Wave Informations Studies (WIS) program. 20, and 50 year return storm waves projected towards the beach were analyzed in order to determine incident wave rays and breaking characteristics. This information can then be used in order to determine the littoral processes present at Narragansett Beach. 
\chapter{Task 1}

\section{Statement of Problem}

\section{Hypotheses and Theories}

\section{Solution of the Problem}

\section{Conclusion}

\newpage
\Hide
\chapter{Task 2}

\section{Statement of Problem}
Task 2 utilizes a wave ray tracing algorithm and local bathymetry information, provided for Narragansett Bay, in order to propagate storm waves determined from task 1. In order to complete this task, simulations are conducted for the 20 and 50 year storms along with creating rays that intersect along the shoreline of Narragansett Beach. For the 20 year storm simulations, use the ray spacing on the beach to calculate a refraction coefficient for each section of beach. Then, using bathymetry data from a chart near the beach, estimate the beach slope. This is used along with the deep water wave properties and refraction coefficient to estimate the shoaling coefficient, breaking depth, and breaker height. 
\section{Hypotheses and Theories}

Wave propagation from deep to shallow water is influenced by changes in the sea floor. As regions of the sea floor change with height the waves refract, and their local angle of incidence shifts. 

\section{Solution of the Problem}

\begin{figure}[H]
\centering
\includegraphics[width=1.0\textwidth]{./img/20y_150deg_hires.eps}
\caption{Wave Ray 20y Expected Wave Parameters at 150 deg Hires}
\label{fig:prob4WHvWD}
\end{figure}

\begin{figure}[H]
\centering
\includegraphics[width=1.0\textwidth]{./img/20y_180deg.eps}
\caption{Wave Ray 20y Expected Wave Parameters at 180 deg}
\label{fig:prob4WHvWD}
\end{figure}

\begin{figure}[H]
\centering
\includegraphics[width=1.0\textwidth]{./img/20y_210deg.eps}
\caption{Wave Ray 20y Expected Wave Parameters at 210 deg}
\label{fig:prob4WHvWD}
\end{figure}


\begin{figure}[H]
\centering
\includegraphics[width=1.0\textwidth]{./img/50y_150deg_hires.eps}
\caption{Wave Ray 50y Expected Wave Parameters at 150 deg}
\label{fig:prob4WHvWD}
\end{figure}

\begin{figure}[H]
\centering
\includegraphics[width=1.0\textwidth]{./img/50y_180deg.eps}
\caption{Wave Ray 50y Expected Wave Parameters at 180 deg}
\label{fig:prob4WHvWD}
\end{figure}

\begin{figure}[H]
\centering
\includegraphics[width=1.0\textwidth]{./img/50y_210deg.eps}
\caption{Wave Ray 50y Expected Wave Parameters at 210 deg}
\label{fig:prob4WHvWD}
\end{figure}

\section{Conclusion}

%--------------------------------Begin References--------------------------------%

%	BibTex Creator: truben.no/latex/bibtex 

\newpage
\nocite{*}
\def\bibindent{1em}
\begin{thebibliography}{1\kern\bibindent}
\makeatletter
\let\old@biblabel\@biblabel
\def\@biblabel#1{\old@biblabel{#1}\kern\bibindent}
\let\old@bibitem\bibitem
\def\bibitem#1{\old@bibitem{#1}\leavevmode\kern-\bibindent}
\makeatother

\bibitem{CEM}
US Army Corps of Engineers (2002). 
$\emph{Coastal Engineering Manual}$

\bibitem{WIS}
US Army Corps of Engineers (2002). 
$\emph{Wave Information Studies Program}$ \\
\url{http://wis.usace.army.mil/}

\bibitem{WIS}
NOAA (2014). 
$\emph{Narragansett bay including newport harbor chart: 13223.}$ \\
\url{http://www.charts.noaa.gov/OnLineViewer/13223.shtml}

\end{thebibliography}

%--------------------------------Begin Appendix--------------------------------%

\begin{appendices}

\chapter{MATLAB Calculations}

%	Matlab code parser block
%-----------------------------------------------------------------
% Must have before any Matlab code

\lstset{language=Matlab, %basicstyle=\color{red},
    breaklines=true, caption={},%
    morekeywords={matlab2tikz},
    basicstyle=\tiny,
    numberstyle=\tiny,
    keywordstyle=\color{blue},%
    morekeywords=[2]{1}, keywordstyle=[2]{\color{black}},
    identifierstyle=\color{black},%
    stringstyle=\color{mylilas},
    commentstyle=\color{mygreen},%
    showstringspaces=false,%without this there will be a symbol in the places where there is a space
    numbers=left,%
    numberstyle={\tiny \color{black}},% size of the numbers
    numbersep=9pt, % this defines how far the numbers are from the text
    emph=[1]{for,end,break},emphstyle=[1]\color{red}, %some words to emphasise
    emph=[2]{word1,word2}, emphstyle=[2]{style},
    captionpos=b,					% sets the caption-position to bottom
}
%-----------------------------------------------------------------
\section{Parser Function}
\lstinputlisting{../"Task 1"/matlab/parser.m}
\section{Task 1 Script}
\lstinputlisting{../"Task 1"/matlab/task1code.m}
\section{Task 2 Waveray Script}
\lstinputlisting{../"Task 2"/matlab/"raytracing"/task2_waveray.m}
\section{Task 2 Breaker Script}
\lstinputlisting{../"Task 2"/matlab/"Breaker Line Analysis"/LATLONDIST.m}

\chapter{Task 1 Appendicies}

\section{Monthly Maximum output example}
\lstinputlisting[caption = Example output from monethextrema\_new.m, label = lst:example150, firstline = 1, lastline = 5] {../"Task 1"/matlab/monthlyExtreme63079_150.txt}

\section{Yearly Average Wave Conditions}

\begin{minipage}{0.36\textwidth}
\begin{table}[H]
	\begin{tabular}{ccc}
\hline
1980.0 & 0.8 & 6.5 \\
1981.0 & 0.8 & 6.5 \\
1982.0 & 0.6 & 6.3 \\
1983.0 & 0.9 & 6.5 \\
1984.0 & 0.9 & 6.7 \\
1985.0 & 0.7 & 6.5 \\
1986.0 & 0.9 & 6.4 \\
1987.0 & 0.9 & 6.3 \\
1988.0 & 0.8 & 6.1 \\
1989.0 & 0.8 & 6.4 \\
1990.0 & 0.9 & 7.0 \\
1991.0 & 0.9 & 6.5 \\
1992.0 & 0.9 & 7.0 \\
1993.0 & 0.8 & 6.7 \\
1994.0 & 0.9 & 6.7 \\
1995.0 & 1.0 & 8.6 \\
1996.0 & 1.2 & 7.3 \\
1997.0 & 0.9 & 6.9 \\
1998.0 & 1.0 & 7.4 \\
1999.0 & 0.9 & 6.9 \\
\hline
\end{tabular}

	\label{yearly150}
	\caption{150$^\circ$ Yearly Averages}
\end{table}
\end{minipage}
\begin{minipage}{0.36\textwidth}
\begin{table}[H]
	\begin{tabular}{ccc}
Year & Avg Hs(m) & Avg Ts (m)\\ \hline
1980.0 & 1.0 & 6.4 \\
1981.0 & 1.0 & 6.6 \\
1982.0 & 1.0 & 6.1 \\
1983.0 & 1.0 & 6.3 \\
1984.0 & 1.1 & 6.6 \\
1985.0 & 0.9 & 6.2 \\
1986.0 & 0.9 & 6.2 \\
1987.0 & 0.9 & 6.1 \\
1988.0 & 0.9 & 6.4 \\
1989.0 & 1.1 & 6.5 \\
1990.0 & 1.0 & 6.8 \\
1991.0 & 1.0 & 6.4 \\
1992.0 & 0.9 & 6.5 \\
1993.0 & 1.0 & 6.5 \\
1994.0 & 1.1 & 6.6 \\
1995.0 & 1.1 & 7.3 \\
1996.0 & 1.1 & 7.2 \\
1997.0 & 1.0 & 7.0 \\
1998.0 & 1.0 & 7.2 \\
1999.0 & 1.0 & 6.8 \\
\hline
\end{tabular}

	\label{yearly180}
	\caption{180$^\circ$ Yearly Averages}
\end{table}
\end{minipage}
\begin{minipage}{0.36\textwidth}
\begin{table}[H]
	\begin{tabular}{ccc}
\hline
1980.0 & 1.1 & 6.4 \\
1981.0 & 1.1 & 6.5 \\
1982.0 & 1.1 & 6.1 \\
1983.0 & 1.0 & 6.0 \\
1984.0 & 0.9 & 5.9 \\
1985.0 & 1.0 & 6.3 \\
1986.0 & 1.0 & 6.3 \\
1987.0 & 0.9 & 6.1 \\
1988.0 & 1.2 & 6.3 \\
1989.0 & 1.1 & 6.4 \\
1990.0 & 1.2 & 6.4 \\
1991.0 & 0.9 & 5.8 \\
1992.0 & 1.0 & 6.1 \\
1993.0 & 1.1 & 6.2 \\
1994.0 & 1.0 & 6.2 \\
1995.0 & 1.1 & 6.3 \\
1996.0 & 1.1 & 6.7 \\
1997.0 & 1.1 & 6.2 \\
1998.0 & 1.0 & 6.2 \\
1999.0 & 1.1 & 6.3 \\
\hline
\end{tabular}

	\label{yearly210}
	\caption{210$^\circ$ Yearly Averages}
\end{table}
\end{minipage}

\end{appendices}
\end{document}